\begin{abstract}
In this thesis we present two approaches to improve automatic playtesting using player modeling. By modeling various cohorts of players we are able to train Convolutional Neural Network based agents that simulate human gameplay using different strategies directly learnt from real player data. The goal is to use the developed agents to predict useful metrics of newly created game content. 

We validated our approaches using the game \textit{Candy Crush Saga}, a non-deterministic match-three puzzle game with a huge search space and more than three thousand levels available.
To the best of our knowledge this is the first time that player modeling is applied in a match-three puzzle game. Nevertheless, the presented approaches are general and can be extended to other games as well. The proposed methods are compared to a baseline approach that simulates gameplay using a single strategy learnt from random gameplay data. 
Results show that by simulating different strategies, our approaches can more accurately predict the level difficulty, measured as the players' success rate, on new levels. Both the approaches improved the mean absolute error by 13\% and the mean squared error by approximately 23\% when predicting with linear regression models. Furthermore, the proposed approaches can provide useful insights to better understand the players and the game.

\vspace{15pt}
\textbf{\textit{Keywords}} --- Player Modeling; Automatic Playtesting; Gameplay Simulation; Convolutional Neural Network. 


\end{abstract}

\clearpage




\begin{foreignabstract}{swedish}
I denna uppsats presenterar vi två tillvägagångssätt för att förbättra automatisk speltestning genom modellering av spelare. Genom att modellera olika grupper av spelare kunde vi träna Convolutional Neural Network-baserade agenter för att simulera mänskligt spelande med hjälp av olika strategier som är lärda direkt från mänsklig spelardata. Målet är att använda de utvecklade agenterna för att förutsäga användbar metrik av nyskapat spelinnehåll.

Vi validerade vårt tillvägagångssätt genom \textit{Candy Crush Saga}, ett icke-deterministiskt 3-matchnings pusselspel med mer än tre tusen nivåer. Detta är första gången som spelarmodellering appliceras på ett 3-matchnings pusselspel. De presenterade tillvägagångssätten är mer generella och kan utökas till andra spel. De föreslagna tillvägagångssätten är jämförda med ett tillvägagångssätt som simulerar spelande genom en strategi som är lärd direkt från slumpmässig mänsklig spelardata. Resultatet visar att vårt tillvägagångssätt, genom simulering av olika strategier är, mer exakt för att förutsäga spelarens svårighet, mätt genom spelarens framgång, på nya nivåer. Båda tillvägagångssätten förbättrade mean absolute error med 13\% och mean squared error med ungefär 23\%. Dessutom kan de föreslagna tillvägagångssätten ge en användbar insikt för att bättre förstå spelarna och spelet.

\end{foreignabstract}